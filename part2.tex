\section{Fitness Weighted Mating}
\par Averaging between males in preparation for mating can be improved by weighing fitness values so that a better gene could be created to be used in mating with a female. The best male lion among suitors in mating will have more influence on the traits of the offsprings.

\par To produce better offsprings, nature has always arranged the better fit organisms to survive. In order to find better offsprings, female organisms would look for better fit organisms among the crowd to mate. To better model this trait in mating between multiple male lions to a female lion. The gene of the best male lion should better influence the gene of the offsprings.

\par \textbf{To simulate this effect, an equation similar to inverse distance weighting \cite{idw} is created that instead uses fitness difference as basis} to create a position that has a weighted average that relies more on better fit positions from multiple males. \cite{idw2}
$$
\text{Lion Average} = \frac{\displaystyle\sum_{i=1}^{n} \left( \frac{\text{Lion}_i}{f_{\text{fem}} - f(\text{Lion}_i)} \right)}{\displaystyle\sum_{i=1}^{n} \left( \frac{1}{f_{\text{fem}} - f(\text{Lion}_i)} \right)}
$$

where $n$ is the number of male lions, Lion$_i$ is the male lion's best position, $f_{\text{fem}}$ is the mating female lion's current fitness and Lion Average is the weighted average of the positions between the male lions based on their fitness.
