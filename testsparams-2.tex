\section{Griewank Function}

\subsection{Two-Dimensional Griewank}

\par The parameter set found to have the best average fitness for the Two-Dimensional Griewank are the parameters

\fbox{\begin{minipage}{0.9\textwidth}
\scriptsize
Nomad Percentage (\%N) = 0.2 \\
Roaming Percentage (\%R) = 0.8 \\
Sex Percentage (\%S) = 0.2 \\
Mating Rate (\%Ma) = 0.6 \\
Mutation Probability = 0.8 \\
Immigration Rate = 0.2 \\
Percent Group Influence = 0.8 \\
Ranked Selection Pressure = 2 \\
Near to Best Random Pressure = 2 \\
\textbf{Parameter Set Average Fitness: 1.1100e-16}
\end{minipage}}

\subsection{Three-Dimensional Griewank}

\par The parameter set found to have the best average fitness for the Three-Dimensional Griewank are the parameters

\fbox{\begin{minipage}{0.9\textwidth}
\scriptsize
Nomad Percentage (\%N) = 0.2 \\
Roaming Percentage (\%R) = 0.6 \\
Sex Percentage (\%S) = 0.2 \\
Mating Rate (\%Ma) = 0.3 \\
Mutation Probability = 0.4 \\
Immigration Rate = 0.8 \\
Percent Group Influence = 0.2 \\
Ranked Selection Pressure = 3 \\
Near to Best Random Pressure = 3 \\
\textbf{Parameter Set Average Fitness: 0.0064}
\end{minipage}}

\par Notice that even using the same function design, the two-dimensional and the three-dimensional Griewank still doesn't have alike parameter sets. This is likely due to the search space of the three-dimensional Griewank being larger than that of the the two-dimensional Griewank. Due to the search space being larger, the three-dimensional function used a less arbitrary point generating parameter set for iLOA so that it could improve only the best solutions when found. 
